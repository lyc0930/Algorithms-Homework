\documentclass{article}
\usepackage[UTF8]{ctex}
\usepackage[T1]{fontenc}
\usepackage[utf8]{inputenc}
\usepackage{float}
\usepackage{placeins}
\usepackage{latexsym}
\usepackage{algorithm}
\usepackage{algorithmic}
\usepackage{amsmath}
\usepackage{amsthm}
\usepackage{tikz}
\usepackage{hyperref}

\hypersetup{
    colorlinks=true,
    linkcolor = black,
    urlcolor=blue!30!green,
}

\title{Homework 1}
\author{PB17000297 罗晏宸}
\date{September 5 2019}

\begin{document}

\maketitle

\section{Exercise 1}
证明:包含$n$个元素的堆的\textsc{Max-Heapify}函数的时间复杂度是$O(logn)$,\textsc{Build-Max-Heap}函
数的时间复杂度是$O(n)$。
\\

\paragraph{解}
\textsc{Max-Heapify}的时间复杂度由递归式
\begin{equation*}
    T(n) \leq T(2n/3) + \Theta(1)
\end{equation*}
给出,要证对某个常数$c > 0$,$T(n) \leq c \cdot \lg{n}$成立。
\begin{proof}
假设此上界对所有正数$m < n$都成立,特别是对于$m =  2n/3$,有$T( 2n/3 ) \leq c \cdot \lg{(2n/3)}$,将其代入递归式,得到
\begin{align*}
    T(n) &\leq c \cdot \lg{\left(\frac{2n}{3}\right)} + \Theta(1) \\
    &\leq c \cdot \lg{n}
\end{align*}
其中,为使最后一步成立,应有
\begin{align*}
    && c \lg{\left(\frac{2n}{3}\right)} + \Theta(1) &\leq c \lg{n}& \\
    \Rightarrow && c \lg{\left( \frac{2}{3} \right)} + \Theta(1) &\leq 0 & \\
    \Rightarrow && c \lg{\left( \frac{3}{2} \right)} &\geq \Theta(1) & \\
    \Rightarrow && c &\geq \frac{\Theta(1)}{\lg{\left( \dfrac{3}{2} \right)}}& \\
\end{align*}
因此对于充分大的$n$,存在常数$c > 0$,使得$T(n) \leq c \cdot \lg{n}$成立,因此$T(n) = O(\lg{n})$。
\end{proof}
一个共有$n$各元素的堆高度为$\lfloor \lg{n} \rfloor$,并且高度为$h$的结点有至多$\left\lceil \dfrac{n}{2^{h+1}} \right\rceil$个,而对于一个高度为$h$的结点,\textsc{Max-Heapify}函数的时间复杂度是$O(logn)=O(h)$,因此\textsc{Build-Max-Heap}的时间复杂度可以由以下证明给出
\begin{proof}
\begin{align*}
    \sum_{h=0}^{\lfloor \lg{n} \rfloor}{\left\lceil \frac{n}{2^{h+1}} \right\rceil \cdot O(h)} &= O\left( n\sum_{h=0}^{\lfloor \lg{n} \rfloor}{\frac{h}{2^h}} \right) \\
    &= O\left( n\sum_{h=0}^{\infty}{\frac{h}{2^h}} \right) \\
    &= O\left( n\sum_{h=0}^{\infty}{\frac{\dfrac{1}{2}}{\left( 1 - \dfrac{1}{2} \right)^2}} \right) \\
    &= O(n \cdot 2) \\
    &= O(n)
\end{align*}
\end{proof}

\section{Exercise 7.2-6 \& 7.2-5}
\subparagraph{(a)}
试证明:在一个随机输入数组上,对于任何常数$0 < \alpha \leq 1/2$,\textsc{Partition}产生比$1−\alpha : \alpha$更平衡的划分的概率约为$1−2\alpha$。
\subparagraph{(b)}
假设快速排序的每一层所做的划分比例都是$1−\alpha : \alpha$,其中$0 < \alpha \leq 1/2$且是一个常数. 试证明:在相应的递归树中,叶结点的最小深度大约是$−\lg{n}/\lg{\alpha}$,最大深度大约是 $−\lg{n}/\lg{(1−\alpha)}$(无需考虑舍入问题)。\\
(注: 堆中结点的\textbf{高度}为该结点\textbf{到叶结点最长简单路径上边的数目};结点的\textbf{深度}为该结点\textbf{到根结点的简单路径上结点的数目})
\\

\paragraph{解}
\subparagraph{(a)}
设随机输入数组$A$有n个数$A_0,\,A_1,\,\cdots,\,A_{\alpha n},\,\cdots,\,A_{(1 - \alpha)n},\,\cdots,\,A_n$。
\begin{proof}
设新的划分为$1 − \beta : \beta$,要使得划分更平衡,应有$|(1 - \beta) - \beta| < 1 - 2\alpha = |(1 - \alpha) - \alpha|$
\begin{itemize}

\item 当$\beta \leq \alpha$时,$|(1 - \beta) - \beta| = 1 - 2\beta \geq 1 - 2\alpha$

\item 当$\beta \geq 1 - \alpha$时,$1 - \beta \geq \alpha$, $|(1 - \beta) - \beta| = 2\beta - 1 \geq 1 - 2\alpha$

\item 当$\alpha < \beta < 1 - \alpha$时,$\alpha < 1 - \beta < 1 - \alpha$,$0<|(1 - \beta) - \beta|<1 - 2\alpha$

\end{itemize}
其中最后一种情况满足要求,假设$\beta$是$[0, \,1]$上的均匀分布,则有
\begin{equation*}
P = P\{\alpha < \beta < 1 - \alpha\}=\dfrac{(1 - \alpha) - \alpha}{1 - 0} = 1 - 2\alpha
\end{equation*}
\end{proof}
\subparagraph{(b)}
对递归式$T(n)=T(\alpha n) + T((1 - \alpha)n) + cn$构造递归树如图\ref{fig:1}所示。
\begin{figure}
    \centering
    \begin{tikzpicture}[
        level 1/.style = {sibling distance = 5cm},
        level 2/.style = {sibling distance = 2cm},
        level 3/.style = {sibling distance = 1cm},
        scale=1.0,
        transform shape]

        \node {$cn$}
            child{node {$c \alpha n$}
                child{node {$c \alpha^2 n$}
                    child[grow = -70]{node {$\vdots$}}
                    child[grow = -110]{node {$\vdots$}}
                    child[grow = left]{node {$-\log_{\alpha}{n}$} edge from parent[draw=none]
                        child [grow = up, level distance = 3.2cm] {node {} edge from parent[->]}
                        child [grow = down, level distance = 1cm] {node {} edge from parent[->]}
                    }
                }
                child{node {$c \alpha(1 - \alpha) n$}
                    child{node {$\vdots$}}
                    child{node {$\vdots$}}
                }
            }
            child{node {$c (1 - \alpha) n$}
                child{node {$c \alpha(1 - \alpha) n$}
                    child{node {$\vdots$}}
                    child{node {$\vdots$}}
                }
                child{node {$c (1 - \alpha)^2 n$}
                    child[grow = -70]{node {$\vdots$}}
                    child[grow = -110]{node {$\vdots$}}
                    child[grow = right, level distance = 2cm]{node {$-\log_{1 - \alpha}{n}$} edge from parent[draw=none]
                        child [grow = right, level distance = 2cm] {node {$cn$} edge from parent[draw = none]
                            child [grow = up, level distance = 1.5cm] {node {$cn$} edge from parent[draw = none]
                                child [grow = up, level distance = 1.5cm] {node {$cn$} edge from parent[draw = none]}
                            }
                            child [grow = down, level distance = 1.5cm] {node {$\vdots$} edge from parent[draw = none]
                                child [grow = down, level distance = 1.5cm] {node {$O(n \lg{n})$} edge from parent[draw = none]}
                            }
                        }
                        child [grow = up, level distance = 3.2cm] {node {} edge from parent[->]}
                        child [grow = down, level distance = 2.5cm] {node {} edge from parent[->]}
                    }
                }
            };
    \end{tikzpicture}
    \caption{\label{fig:1}为表达式式$T(n)=T(\alpha n) + T((1 - \alpha)n) + cn$构造递归树}

\end{figure}
\begin{proof}
从递归树的根到叶结点最右和最左的简单路径长度分别为$-\log_{1 - \alpha}{n}$与$-\log_{\alpha}{n}$,
\begin{align*}
    && 0 < \alpha &\leq 1/2 & \\
    \Rightarrow && \alpha &\leq  1 - \alpha & \\
    \Rightarrow && \lg{\alpha} &\leq \lg{(1 - \alpha)} & \\
    \Rightarrow && -\dfrac{1}{\lg{\alpha}} &\leq -\dfrac{1}{\lg{(1 - \alpha)}} & \\
    \Rightarrow && -\dfrac{\lg{n}}{\lg{\alpha}} &\leq -\dfrac{\lg{n}}{\lg{(1 - \alpha)}} & \\
    \Rightarrow && -\log_{\alpha}{n} &\leq -\log_{1 - \alpha}{n} &
\end{align*}
对于两者之间其他任何一个叶结点的深度$l$,有
\begin{align*}
    && \alpha^k(1 - \alpha)^{l-k}n &\leq 1 ,& 0 < k \leq l \\
    \Rightarrow && k + (l-k)\log_{\alpha}{(1 - \alpha)} + \log_{\alpha}{n} &\leq 0 ,& 0 < k \leq l \\
    \Rightarrow && l &\geq -\dfrac{\log_{\alpha}{n} + k}{\log_{\alpha}{(1 - \alpha)}}+k ,& 0 < k \leq l \\
    \Rightarrow && l &\geq -(\log_{\alpha}{n} + k) - k ,& 0 < k \leq l \\
    \Rightarrow && l &\geq -\log_{\alpha}{n} & \\
\end{align*}
同理可证$l \leq -\log_{1 - \alpha}{n}$。因此叶结点的最小深度大约是$-\log_{\alpha}{n} = −\lg{n}/\lg{\alpha}$,最大深度大约是 $-\log_{1 - \alpha}{n} = −\lg{n}/\lg{(1−\alpha)}$
\end{proof}

\section{OnlineJudge Problem H3-1 数字统计}

\paragraph{解}
\href{https://202.38.86.171/status/ce8c27ff6b1ffab6e9fb4db6dd770a60}{\underline{Accepted}}
\\

\section{OnlineJudge Problem H3-2 考试排名}

\paragraph{解}
\href{https://202.38.86.171/status/c68c59682ab2efe4a7fd628089e117c7}{\underline{Accepted}}
\\

\end{document}