\documentclass{article}
\usepackage[UTF8]{ctex}
\usepackage[T1]{fontenc}
\usepackage[utf8]{inputenc}
\usepackage{float}
\usepackage{placeins}
\usepackage{latexsym}
\usepackage{algorithm}
\usepackage{algorithmic}
\usepackage{amsmath}
\usepackage{amsthm}
\title{Homework 1}
\author{PB17000297 罗晏宸}
\date{September 5 2019}

\begin{document}

\maketitle

\section{Exercise 1}
证明:包含$n$个元素的堆的\textsc{Max-Heapify}函数的时间复杂度是$O(logn)$,\textsc{Build-Max-Heap}函
数的时间复杂度是$O(n)$。
\\

\paragraph{解}
\\

\section{Exercise 2}
\subparagraph{(a)}
试证明:在一个随机输入数组上,对于任何常数$0 < \alpha \leq 1/2$,\textsc{Partition}产生比$1−\alpha : \alpha$更平衡的划分的概率约为$1−2\alpha$。
\subparagraph{(b)}
假设快速排序的每一层所做的划分比例都是$1−\alpha : \alpha$,其中$0 < \alpha \leq 1/2$且是一个常数. 试证明:在相应的递归树中,叶结点的最小深度大约是$−\lg{n}/\lg{\alpha}$,最大深度大约是 $−\lg{n}/\lg{(1−\alpha)}$(无需考虑舍入问题)。\\
(注: 堆中结点的\textbf{高度}为该结点\textbf{到叶结点最长简单路径上边的数目};结点的\textbf{深度}为该结点\textbf{到根结点的简单路径上结点的数目})
\\

\paragraph{解}
\subparagraph{(a)}
\subparagraph{(b)}
\\

\section{OnlineJudge Problem H3-1 数字统计}
\\

\section{OnlineJudge Problem H3-2 考试排名}
\\
\end{document}