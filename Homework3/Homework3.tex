\documentclass{article}
\usepackage[UTF8]{ctex}
\usepackage[T1]{fontenc}
\usepackage[utf8]{inputenc}
\usepackage{float}
\usepackage{placeins}
\usepackage{latexsym}
\usepackage{algorithm}
\usepackage{algorithmic}
\usepackage{amsmath}
\usepackage{amsthm}
\title{Homework 1}
\author{PB17000297 罗晏宸}
\date{September 5 2019}

\begin{document}

\maketitle

\section{Exercise 1}
证明:包含$n$个元素的堆的\textsc{Max-Heapify}函数的时间复杂度是$O(logn)$,\textsc{Build-Max-Heap}函
数的时间复杂度是$O(n)$。
\\

\paragraph{解}
\textsc{Max-Heapify}的时间复杂度由递归式
\begin{equation*}
    T(n) \leq T(2n/3) + \Theta(1)
\end{equation*}
给出,要证对某个常数$c > 0$,$T(n) \leq c \cdot \lg{n}$成立。
\begin{proof}
假设此上界对所有正数$m < n$都成立,特别是对于$m =  2n/3$,有$T( 2n/3 ) \leq c \cdot \lg{(2n/3)}$,将其代入递归式,得到
\begin{align*}
    T(n) &\leq c \cdot \lg{\left(\frac{2n}{3}\right)} + \Theta(1) \\
    &\leq c \cdot \lg{n} 
\end{align*}
其中,为使最后一步成立,应有
\begin{align*}
    && c \lg{\left(\frac{2n}{3}\right)} + \Theta(1) &\leq c \lg{n}& \\
    \Rightarrow && c \lg{\left( \frac{2}{3} \right)} + \Theta(1) &\leq 0 & \\
    \Rightarrow && c \lg{\left( \frac{3}{2} \right)} &\geq \Theta(1) & \\
    \Rightarrow && c &\geq \frac{\Theta(1)}{\lg{\left( \frac{3}{2} \right)}}& \\
\end{align*}
因此对于充分大的$n$,存在常数$c > 0$,使得$T(n) \leq c \cdot \lg{n}$成立,因此$T(n) = O(\lg{n})$。
\end{proof}
\textsc{Build-Max-Heap}的时间复杂度由递归式
\begin{equation*}
    
\end{equation*}
\\

\section{Exercise 2}
\subparagraph{(a)}
试证明:在一个随机输入数组上,对于任何常数$0 < \alpha \leq 1/2$,\textsc{Partition}产生比$1−\alpha : \alpha$更平衡的划分的概率约为$1−2\alpha$。
\subparagraph{(b)}
假设快速排序的每一层所做的划分比例都是$1−\alpha : \alpha$,其中$0 < \alpha \leq 1/2$且是一个常数. 试证明:在相应的递归树中,叶结点的最小深度大约是$−\lg{n}/\lg{\alpha}$,最大深度大约是 $−\lg{n}/\lg{(1−\alpha)}$(无需考虑舍入问题)。\\
(注: 堆中结点的\textbf{高度}为该结点\textbf{到叶结点最长简单路径上边的数目};结点的\textbf{深度}为该结点\textbf{到根结点的简单路径上结点的数目})
\\

\paragraph{解}
\subparagraph{(a)}
\subparagraph{(b)}
\\

\section{OnlineJudge Problem H3-1 数字统计}
\\

\section{OnlineJudge Problem H3-2 考试排名}
\\
\end{document}