\documentclass{article}
\usepackage[UTF8]{ctex}
\usepackage[T1]{fontenc}
\usepackage[utf8]{inputenc}
\usepackage{latexsym}
\usepackage{amsmath}
\usepackage{amsthm}
\usepackage{amssymb}
\usepackage{hyperref}
\usepackage{clrscode3e}
\usepackage{xcolor}

\newcommand{\SetHrefColor}[1]{
	\hypersetup{urlcolor=#1}
}

\hypersetup{
	colorlinks=true,
	urlcolor=black,
}

\title{Homework 6}
\author{PB17000297 罗晏宸}
\date{October 26 2019}



\begin{document}

\maketitle

\section{Problem 14-1 Point of maximum overlap}
假设我们希望记录一个区间集合的最大重叠点,即被最多数目区间所覆盖的那个点。
\subparagraph*{a}
证明:最大重叠点一定是其中一个区间的端点
\subparagraph*{b}
设计一个数据结构,使得它能够有效地支持\textsc{Interval-Insert}、\\\textsc{Interval-Delete},以及返回最大重叠点的\textsc{Find-Pom}操作。
\paragraph{解}
\subparagraph*{a}
假设所有的 最大重叠点 均不是任意区间的端点,设一个 最大重叠点 为$m$,与其相距最近的 区间 低端点 为$i.low$,与其相距最近的 区间 高端点为$j.high$,则在区间$[i.low, j.high]$上的所有点均被和$m$相同数目的区间所覆盖,故$i.low$与$j.high$也是 最大重叠点 ,这与假设矛盾,因此假设不成立,即最大重叠点中一定有区间的端点。
\subparagraph*{b}
维护一个升序链表,链表的基类型是一个拥有两个成员的结构体,成员$i$表示区间的序号(加入数组的次序),成员$end$表示区间的端点位置,即对于每一个区间$x$,数组中有两个元素分别对应其左端点和右端点。这个链表的有序基于对于各成员端点值的比较,通过二分查找定位,\textsc{Interval-Insert}、\textsc{Interval-Delete}操作均可以在$O(\lg{n})$的时间下实现。对于\textsc{Find-Pom}操作,从表头开始遍历链表,同时对每个区间,记录其左端点是否已出现且仅出现左端点(即对于每一个序号$i$,记录其出现次数是否为1),对于每一个左端点,覆盖厚度加一,每出现一个右端点(区间的端点此前出现过),覆盖厚度减一,比较是否更新最大覆盖厚度,且更新时记录端点,最终返回最大覆盖厚度对应的端点,以上操作消耗的时间是$O(n)$。

\section{Problem 19-1 Alternative implementation of deletion}
Pisano教授提出了下面的 \textsc{Fib-Heap-Delete}过程的一个变种,声称如果删除的结点不是由$H.min$指向的结点,那么该程序运行地更快。
\begin{codebox}
\Procname{$\proc{Pisano-Delete}(H,x)$}
\li	\If $x \isequal \attrib{H}{min}$ \Then
\li		$\proc{Fib-Heap-Delete}(H)$
\li \Else
\li		$y \gets \attrib{x}{p}$
\li		\If $y \neq \const{nil}$ \Then
\li			$\proc{Cut}(H,x,y)$
\li			$\proc{Cascading-Cut}(H,y)$
		\End
\li		add $x$'s child list to the root list of $H$
\li		remove $x$ from the root list of $H$
	\End
\end{codebox}
\subparagraph*{a}
该教授的声称是基于第8行可以在$O(1)$实际时间完成的这一假设,它的程序可以运行的更快。该假设有什么问题吗?
\subparagraph*{b}
当$x$不是由$H.min$指向时,给出\textsc{Pisano-Delete}实际时间的一个好(紧凑)上界。你给出的上界应该以$x.degree$和调用\textsc{Cascading-Cut}的次数$c$这两个参数来表示。

\paragraph{解}
\subparagraph*{a}
该假设的问题在于,将$x$的所有孩子添加到$H$中需要的时间应当依赖于$x$孩子的个数,并不是常数时间。
\subparagraph*{b}
由上题可知,第8行实际时间应为$O(x.degree)$,而\textsc{Cut}和\\
\textsc{Cascading-Cut}是常数时间内可以完成的,因此当$x$不是由$H.min$指向时,\textsc{Pisano-Delete}实际时间是$O(x.degree+c)$。

\section{Exercise 21.2-1}
使用链表表示和加权合并启发式策略,写出\textsc{Make-Set},\textsc{Find-Set}和\textsc{Union}操作的伪代码。

\paragraph{解}
\begin{codebox}
\Procname{$\proc{Make-Set}(x)$}
\li	Let o be an object with three fields, next, value, and set
\li	Let L be a linked list object with head = tail = o
\li $\attrib{o}{next} \gets \const{nil}$
\li $\attrib{o}{set} \gets L$
\li $\attrib{o}{value} \gets x$
\li	$\attrib{L}{size} \gets 1$
\li \Return $L$
\end{codebox}

\begin{codebox}
\Procname{$\proc{Find-Set}(x)$}
\li	\Return $\attribbb{o}{set}{head}{value}$
\end{codebox}

\begin{codebox}
\Procname{$\proc{Union}(x,y)$}
\li \If $\attribb{x}{set}{size} \geq \attribb{y}{set}{size}$ \Then
\li 	$L1 \gets \attrib{x}{set}$
\li 	$L2 \gets \attrib{y}{set}$
\li	\Else
\li 	$L1 \gets \attrib{y}{set}$
\li 	$L2 \gets \attrib{x}{set}$
	\End
\li	$\attribb{L1}{tail}{next} \gets \attrib{L2}{head}$
\li $z \gets \attrib{L2}{head}$
\li \While $\attrib{z}{next} \neq \const{nil}$
	\Do
\li		$\attrib{z}{set} \gets L1$
	\End
\li	$\attrib{L1}{tail} \gets \attrib{L2}{tail}$
\li $\attrib{L1}{size} \gets \attrib{L1}{size} + \attrib{L2}{size}$
\li \Return $L1$
\end{codebox}
\section{\href{https://202.38.86.171/problem/H6-1}{OnlineJudge Problem H6-1 在线比赛}}

\SetHrefColor{blue!30!green}

\paragraph{解}
\href{https://202.38.86.171/status/829aee86f921ebd694c0d52303582a55}{\underline{Accepted}}

\SetHrefColor{black}
\section{\href{https://202.38.86.171/problem/H6-1}{OnlineJudge Problem H6-2 朋友圈}}
\SetHrefColor{blue!30!green}
\paragraph{解}
\href{https://202.38.86.171/status/7ea7548feef36a762b4a53e940d47aba}{\underline{Accepted}}


\end{document}