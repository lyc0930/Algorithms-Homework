\documentclass{article}
\usepackage[UTF8]{ctex}
\usepackage[T1]{fontenc}
\usepackage[utf8]{inputenc}
\usepackage{latexsym}
\usepackage{amsmath}
\usepackage{amsthm}
\usepackage{amssymb}
\usepackage{hyperref}
\usepackage{clrscode3e}
\usepackage{xcolor}

\newcommand{\SetHrefColor}[1]{
	\hypersetup{urlcolor=#1}
}

\hypersetup{
	colorlinks=true,
	urlcolor=black,
}

\title{Homework 7}
\author{PB17000297 罗晏宸}
\date{November 12 2019}



\begin{document}

\maketitle

\section{Problem 15-1 Longest simple path in a directed acyclic graph}
给定 一个 有向无环图 $G = (V,E)$ , 边 权重 为 实数 , 给定 图中 两个 顶点 $s$ 和 $t$ 。 设计 动态规划 算法 , 求 从 $s$ 到 $t$ 的 最长 加权 简单 路径 。

\paragraph{解}

\section{Exercise 16.2-2}
设定 动态规划 算法 求解 0-1 背包问题 , 要求 运行时间 为 $O(nW)$,$n$ 为 商品 数量 , $W$ 是 小偷 能 放进 背包 的 最大 商品 总 重量 。

\paragraph{解}


\section{Problem 15-6 Planning a company party}
一位 公司 主席 正在 向 Stewart 教授 咨询 公司 聚会 方案 。 公司 的 内部 结构 关系 是 层次化 的 , 即 员工 按 主管-下属 关系 构成 一 棵 树 , 根 结点 为 公司 主席 。 人事部 按 “ 宴会 交际 能力 ” 为 每个 员工 打分 , 分值 为 实数 。 为了 使 所有 参加 聚会 的 员工 都 感到 愉快 , 主席 不 希望 员工 及其 直接 主管 同时 出席 。
\par
公司 主席 向 Stewart 教授 提供 公司 结构 树 , 采用 左 孩子 右 兄弟 表示法 ( 参见 课本 10.4 节 ) 描述 。 每 个 节点 除了 保存 指针 外 , 还 保存 员工 的 名字 和 宴会 交际 评分 。 设计 算法 , 求 宴会 交际 评分 之 和 最大 的 宾客 名单 。 分析  算法 复杂度 。

\paragraph{解}

\section{Exercise 16.2-5}
设计 一 个 高效 的 算法 , 对 实数线 上 给定 的 一 个 点集 ${x_1,x_2,...,x_n}$ , 求 一 个 单位 长度 闭 区间 的 集合 , 包含 所有 给定 的 点 , 并 要求 此 集合 最小 。 证明 你 的 算法 是 正确 的 。

\paragraph{解}

\section{Problem 16-1 a. Coin changing }
考虑 用 最少 的 硬币 找 $n$ 美分 零钱 的 问题 。 假定 每 种 硬币 的 面额 都 是 整数 。 设计 贪心 算法 求解 找零 问题 , 假定 有 25 美分 、 10 美分 、 5 美分 和 1 美分 四 种 面额 的 硬币 。 证明 你 的 算法 能 找到 最优解 。

\paragraph{解}

\end{document}