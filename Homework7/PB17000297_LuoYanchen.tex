\documentclass{article}
\usepackage[UTF8]{ctex}
\usepackage[T1]{fontenc}
\usepackage[utf8]{inputenc}
\usepackage{latexsym}
\usepackage{amsmath}
\usepackage{amsthm}
\usepackage{amssymb}
\usepackage{hyperref}
\usepackage{clrscode3e}
\usepackage{xcolor}

\newcommand{\SetHrefColor}[1]{
	\hypersetup{urlcolor=#1}
}

\hypersetup{
	colorlinks=true,
	urlcolor=black,
}

\title{Homework 7}
\author{PB17000297 罗晏宸}
\date{November 12 2019}



\begin{document}

\maketitle

\section{Problem 15-1 Longest simple path in a directed acyclic graph}
给定 一个 有向 无环 图 $G = (V,E)$ , 边 权重 为 实数 , 给定 图中 两个 顶点 $s$ 和 $t$ 。 设计 动态规划 算法 , 求 从 $s$ 到 $t$ 的 最长 加权 简单 路径 。

\paragraph{解}

由于 $G$ 是有向无环图,且算法要求的是一条简单路径。因此不需要考虑回路,记$W(uv)$是边$uv$的权重,记$F(p,q)$ 是在图$G$中从$p$到$q$的最长加权简单路径,给出转移方程如下

\begin{equation}
	L(p,q) = \max_{pp' \in E}{ \{ W(p,p') + L(p',q)\} }
\end{equation}

\section{Exercise 16.2-2}
设定 动态规划 算法 求解 0-1 背包问题 , 要求 运行时间 为 $O(nW)$,$n$ 为 商品 数量 , $W$ 是 小偷 能 放进 背包 的 最大 商品 总 重量 。

\paragraph{解}
设$F[i,w]$表示前$i$件物品恰放入一个容量为$w$的背包可以获得的最大价值,给出状态转移方程:
\begin{equation}
	F[i,w] = \max{\{ F[i-1,w],\,F[i-1,w-w_i] + v_i \}}
\end{equation}
其中$w_i$与$v_i$分别是第$i$件物品的重量和价值。
\par
据此给出算法伪代码如下:
\begin{codebox}
	\Procname{$\proc{0-1 Knapsack}(n,W,w,v)$}
	\li $F[0, 0..W] \gets 0$
	\li \For $i \gets 1$ \To $n$
		\Do
	\li 	\For $w \gets w_i$ \To $W$
			\Do
	\li			$F[i,w] \gets \max{\{ F[i-1,w],\,F[i-1,w-w_i] + v_i \}}$
			\End
		\End
	\li \Return $F[n,W]$
\end{codebox}

\section{Problem 15-6 Planning a company party}
一位 公司 主席 正在 向 Stewart 教授 咨询 公司 聚会 方案 。 公司 的 内部 结构 关系 是 层次化 的 , 即 员工 按 主管-下属 关系 构成 一 棵 树 , 根 结点 为 公司 主席 。 人事部 按 “ 宴会 交际 能力 ” 为 每个 员工 打分 , 分值 为 实数 。 为了 使 所有 参加 聚会 的 员工 都 感到 愉快 , 主席 不 希望 员工 及其 直接 主管 同时 出席 。
\par
公司 主席 向 Stewart 教授 提供 公司 结构 树 , 采用 左 孩子 右 兄弟 表示法 ( 参见 课本 10.4 节 ) 描述 。 每 个 节点 除了 保存 指针 外 , 还 保存 员工 的 名字 和 宴会 交际 评分 。 设计 算法 , 求 宴会 交际 评分 之 和 最大 的 宾客 名单 。 分析  算法 复杂度 。

\paragraph{解}
由于主席不希望员工及其直接主管同时出席,对于结构树上的一个结点,若将其加入宾客名单,则不能将其左孩子及其左孩子的全部右兄弟加入名单。设$T$表示公司结构树,其属性$root$对应公司主席、$number$对应全公司人数。$C[i]$表示结构树上以结点$i$为根的子树的最大宴会交际评分之和,其中$i$是结点在$T$中的后序遍历序号,记$GuestList[i]$是$C[i]$对应的宾客名单。
\begin{codebox}
	\Procname{$\proc{Company Party Plan}(T)$}
	\li \For $i \gets 1$ \To $\attrib{T}{number}$
		\Do
	\li \If $\attrib{i}{left-child} \isequal \const{nil}$ \Then
	\li 	$C[i] \gets \attrib{i}{conviviality}$
	\li 	$GuestList[i] \gets \{ \attrib{i}{name} \}$
	\li \Else
	\li 	$C[i] \gets \max{\left \{\displaystyle \sum_{\attrib{j}{parent} = i}{C[j]},\,\sum_{\attribb{j}{parent}{parent} = i}{C[j]} + \attrib{i}{conviviality} \right \}}$
	\li 	add $\arg\max{\Big \{ \{j | \attrib{j}{parent} = i\},\,\{j | \attribb{j}{parent}{parent} = i\}\Big \}}$ into
	\zi		$GuestList[i]$
		\End
		\End
	\li \Return $GuestList[T.root]$
\end{codebox}
由于在循环中需要向前寻找$i$的孩子和孙子,所以算法的时间复杂度是$O(n^2)$的。

\section{Exercise 16.2-5}
设计 一 个 高效 的 算法 , 对 实数线 上 给定 的 一 个 点集 ${x_1,x_2,\cdots,x_n}$ , 求 一 个 单位 长度 闭 区间 的 集合 , 包含 所有 给定 的 点 , 并 要求 此 集合 最小 。 证明 你 的 算法 是 正确 的 。

\paragraph{解}
本题可以用贪心的思想解决,对于给定的点集,设其中最小的点是$x_{min}$,则在给出的单位长度闭区间集合${s_1,s_2,\cdots,s_m}$(假设按左端点从小到大排列)中,在一个最优解中一定有$x_{min} \in s_1$,且$s_1 = [x_{min}, x_{min} + 1]$,因为若$\nexists x \in (x_{min}, x_{min} + 1]$,则在$[x_{min} - 1, x_{min} + 1]$中的任意单位长度闭区间或$s_1$对于最后的集合元素个数都没有影响,否则$s_1$包含的$[x_{min}, x_{min} + 1]$中的点一定不少于任何单位长度闭区间。则问题可以递归的转化为逐点贪心的过程,得到的集合是最优的。设$\mathcal{I}[X]$表示包含点集$X$中所有点的最小单位长度闭区间个数,则有:
\begin{equation}
	\mathcal{I}[X] =
	\left\{
	\begin{aligned}
		&1 + \mathcal{I}\left[X \Big/ \big\{ x|x \in [x_{min}, x_{min} + 1] \big\} \right], &X \neq \varnothing \\
		&0, &X = \varnothing
	\end{aligned}
	\right.
\end{equation}
\par
据此给出算法伪代码如下:
\begin{codebox}
	\Procname{$\proc{Unit-Length Closed Intervals Contain Set}(X)$}
	\li sort $X$
	\li \While $X \neq \varnothing$
		\Do
	\li 	add $\big[ X[0], X[0] + 1 \big]$ to the set
	\li		$X \gets X \Big/ \Big\{ x|x \in \big[X[0], X[0] + 1 \big] \Big\}$
		\End
	\li \Return the set
\end{codebox}

\section{Problem 16-1 a. Coin changing }
考虑 用 最少 的 硬币 找 $n$ 美分 零钱 的 问题 。 假定 每 种 硬币 的 面额 都 是 整数 。 设计 贪心 算法 求解 找零 问题 , 假定 有 25 美分 、 10 美分 、 5 美分 和 1 美分 四 种 面额 的 硬币 。 证明 你 的 算法 能 找到 最优解 。

\paragraph{解}
认为最优解即使得找零的硬币最少,为达到这一目的,每次寻找能找的最大面额的硬币,算法用伪代码可表述如下:
\begin{codebox}
	\Procname{$\proc{Coin Changing}(n)$}
	\li $quarter \gets 0$
	\li $dime \gets 0$
	\li $nickel \gets 0$
	\li $penny \gets 0$
	\li \While $n \geq 25$
		\Do
	\li		$quarter \gets quarter + 1$
	\li		$n \gets n - 25$
		\End
	\li \While $n \geq 10$
		\Do
	\li		$dime \gets dime + 1$
	\li		$n \gets n - 10$
		\End
	\li \While $n \geq 5$
		\Do
	\li		$nickel \gets nickel + 1$
	\li		$n \gets n - 5$
		\End
	\li $penny \gets n$
\end{codebox}
假设这不是最优解,即存在这样的解${quarter',dime',nickel',penny'}$,使得
\begin{align*}
	& n \\
	=& 25quarter' + 10dime' + 5nickel' + penny' \\
	=& 25quarter + 10dime + 5nickel + penny \\
	&quarter' + dime' + nickel' + penny' < quarter + dime + nickel + penny
\end{align*}
记$\Delta q = quarter' - quarter,\Delta d = dime' - dime,\Delta n = nickel' - nickel,\Delta p = penny' - penny$,则有
\begin{align*}
	\left\{
	\begin{aligned}
		\Delta q + \Delta d + \Delta n + \Delta p &< 0 \\
		25\Delta q + 10\Delta d + 5\Delta n + \Delta p &= 0
	\end{aligned}
	\right.
\end{align*}
若$\Delta q < 0$则必有$10\Delta d + 5\Delta n + \Delta p > 25|\Delta q| > 0$,同理若$\Delta d < 0$则必有$5\Delta n + \Delta p > 10|\Delta d| > 0$,若$\Delta n < 0$则必有$\Delta p > 5|\Delta n| > 0$,因此$\Delta q = \Delta d = \Delta n = \Delta p = 0$,即两解相同,矛盾,因此算法得到的是最优解。
\end{document}