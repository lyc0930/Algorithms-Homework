\documentclass{article}
\usepackage[UTF8]{ctex}
\usepackage[T1]{fontenc}
\usepackage[utf8]{inputenc}
\usepackage{float}
\usepackage{placeins}
\usepackage{latexsym}
\usepackage[plain]{algorithm}
\usepackage{algorithmic}
\usepackage{amsmath}
\usepackage{amsthm}
\usepackage{amssymb}
\usepackage{tikz}
\usepackage{hyperref}

% \floatname{algorithm}{}

\hypersetup{
    colorlinks=true,
    linkcolor = black,
    urlcolor=blue!30!green,
}

\title{Homework 5}
\author{PB17000297 罗晏宸}
\date{October 19 2019}

\begin{document}
\maketitle

\section*{Exercise 1}
考虑一个大小为$m = 1000$的散列表和一个对应的散列函数$h(k) = \lfloor m(kA\ \text{mod}\ 1) \rfloor$,其中$A = (\sqrt{5} − 1)/2$,试计算关键字61, 62, 63, 64和65被映射到的位置。
\paragraph{解}

\section*{Exercise 2}
考虑用开放寻址法将关键字10, 22, 31, 4, 15, 28, 17, 88, 59插入到一长度为$m = 11$的散列表中,辅助散列函数为$h'(k) = k$。试说明分别用线性探查、二次探查($c_1 = 1,\,c_2 = 3$)和双重散列($h_1(k) = k,\,
h_2(k) = 1 + (k\ \text{mod}\ (m − 1))$)将这些关键字插入散列表的过程。

\paragraph{解}

\section*{Exercise 3}
因为在基于比较的排序模型中,完成$n$个元素的排序,其最坏情况下需要$\Omega(n \lg{n})$时间。试证明:任何基于比较的算法从$n$个元素的任意序列中构造一棵二叉搜索树,其最坏情况下需要$\Omega(n \lg{n})$的时间。

\paragraph{解}

\section*{Exercise 4}
\subsection*{(a)}
将关键字41, 38, 31, 12, 19, 8连续地插入一棵初始为空的红黑树之后,试画出该结果树。
\subsection*{(b)}
对于(a)中得到的红黑树,依次删除8, 12, 19,试画出每次删除操作后的红黑树。
\paragraph{解}

\end{document}