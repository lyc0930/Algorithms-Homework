\documentclass{article}
\usepackage[UTF8]{ctex}
\usepackage[T1]{fontenc}
\usepackage[utf8]{inputenc}
\usepackage{latexsym}
\usepackage{amsmath}
\usepackage{amsthm}
\usepackage{amssymb}
\usepackage{clrscode3e}
\usepackage{xcolor}

\title{Homework 10}
\author{PB17000297 罗晏宸}
\date{December 26 2019}

\begin{document}

\maketitle

\section{Exercise 34.5-1}
\textbf{子图同构问题}取两个无向图 $G_1$ 和 $G_2$, 要回答 $G_1$ 是否与 $G_2$ 的一个子图同构这一问题。证明:子图同构问题是 NP 完全的。

\paragraph{解}


\section{Exercise 34.5-6}
证明:哈密顿路径问题是 NP 完全的。

\paragraph{解}


\section{Exercise 35.2-4}
在\textbf{瓶颈旅行商问题}中,目标是找出这样一条哈密顿回路,使得回路中代价最大的边的代价相对于其他回路来说最小。假设代价函数满足三角不等式,证明:这个问题存在一个近似比为 3 的多项式时间近似算法。

\paragraph{解}


\section{Problem 35-6 Approximating a maximum spanning tree}
设 $G = (V, E)$ 是一个无向图,其中的每条边 $(u, v) \in E$ 具有不同的权值 $w(u,v)$。对每个顶点 $v \in V$,设 $\max{(v)} = \displaystyle \arg{\max_{(u,v) \in E}{\{ w(u, v) \}}}$ 是与顶点 $v$ 相关联的最大权值边。设 $S_G = \{ \max{(v)} : v \in V\} $ 表示与各个顶点相关联的最大权值边的集合,$T_G$ 表示图 $G$ 的最大权值生成树。对任意的边集 $E' \subseteq E$,定义 $w(E') = \displaystyle \sum_{(u,v) \in E'}{w(u,v)}$。
\subparagraph{a} 给出一个至少包含 4 个顶点的图,使其满足 $S_G = T_G$。
\subparagraph{b} 给出一个至少包含 4 个顶点的图,使其满足 $S_G \neq T_G$。
\subparagraph{c} 证明:对任意的图 $G$,$S_G \subseteq T_G$。
\subparagraph{d} 证明:对任意的图 $G$,$w(T_G) \geq w(S_G) / 2$。
\subparagraph{e} 给出一个 $O(V + E)$ 时间算法,用于计算 2 近似的最大生成树。


\paragraph{解}
\subparagraph{a}
\subparagraph{b}
\subparagraph{c}
\subparagraph{d}
\subparagraph{e}

\section{Exercise 34.4-7}
给出一种 2-SAT 问题的多项式解法
\paragraph{解}

\end{document}