\documentclass{article}
\usepackage[UTF8]{ctex}
\usepackage[T1]{fontenc}
\usepackage[utf8]{inputenc}
\usepackage{float}
\usepackage{placeins}
\usepackage{latexsym}
\usepackage{amsmath}
\usepackage{amsthm}
\usepackage{verbatim}
\usepackage{fancybox}
\usepackage{hyperref}
\usepackage{tikz}

\hypersetup{
    colorlinks=true,
    linkcolor = black,
    urlcolor=blue!30!green,
}

\title{Homework 2}
\author{PB17000297 罗晏宸}
\date{September 13 2019}

\begin{document}

\maketitle

\section{Exercise 3.1-4}
$2^{n+1}=O(2^n)$成立吗?$2^{2n}=O(2^n)$成立吗?
\\

\paragraph{解}
前者成立,但后者不成立。\par
取$c=2$与$n_0=1$,有
\begin{equation*}
    0 \leq 2^{n+1} = 2 \cdot 2^n \leq c \cdot 2^n, \qquad \forall n \geq n_0 = 1
\end{equation*}
因此$2^{n+1}=O(2^n)$。\par
假设存在常数$c$与$n_0$使得
\begin{equation*}
    0 \leq 2^{2n} \leq c \cdot 2^n, \qquad \forall n \geq n_0
\end{equation*}
成立,则有
\begin{align*}
    && 0 \leq 2^{2n} &\leq c \cdot 2^n, & \forall n \geq n_0 \\
    \Rightarrow && 2^n \cdot 2^n &\leq c \cdot 2^n, &  \forall n \geq n_0 \\
    \Rightarrow && 2^n &\leq c, & \forall n \geq n_0
\end{align*}
这与$c$是常数矛盾!因此假设不成立,$2^{2n} \neq O(2^n)$。
\\

\section{Exercise 3.2-3}
证明等式
\begin{equation}
    \lg{(n!)}=\Theta(n \lg{n}) \tag{3.19}
\end{equation}
并证明$n!=\omega (2^n)$且$n!=o(n^n)$。
\\

\paragraph{解}
取常数$c_1 = 1 - \dfrac{\lg{e}}{2}$、$c_2 = 1$和$n_0 = 4$,有
\begin{align*}
                && n!      &\leq n^n, & \forall n \geq 1 \\
    \Rightarrow && \lg{n!} &\leq \lg{n^n},                    & \forall n \geq 1 \\
    \Rightarrow && \lg{n!} &\leq n \lg{n} = 1 \cdot n \lg{n}, & \forall n \geq 1 \\
    \Rightarrow && \lg{n!} &\leq c_2 \cdot n \lg{n}, & \forall n \geq n_0 > 1 \\
    \\
                && c_1 = 1 - \frac{\lg{e}}{2} &\leq 1 - \dfrac{\lg{e}}{n}, & \forall n \geq 4 \\
    \Rightarrow && \frac{\lg{e}}{\lg{n}} &\leq 1 - c_1, & \forall n \geq 4 \\
    \Rightarrow && \lg{e} &\leq (1-c_1)\lg{n}, & \forall n \geq 4 \\
    \Rightarrow && n\lg{e} &\leq n \lg{n} - c_1 \cdot n\lg{n}, & \forall n \geq 4 \\
    \Rightarrow && c_1 \cdot n \lg{n} &\leq n\lg{n}-n\lg{e}, & \forall n \geq 4 \\
    \Rightarrow && c_1 \cdot n \lg{n} &\leq \lg{\left(\frac{n}{e}\right)^n}, & \forall n \geq 4 \\
    \Rightarrow && c_1 \cdot n \lg{n} &\leq \lg{\left[\sqrt{2 \pi n}\left(\frac{n}{e}\right)^n\right]}, & \forall n \geq 4 \\
    \Rightarrow && c_1 \cdot n \lg{n} &\leq \lg{\left[\sqrt{2 \pi n}\left(\frac{n}{e}\right)^n \left(1+\Theta \left(\frac{1}{n}\right) \right)\right]}, & \forall n \geq 4 \\
    \Rightarrow && c_1 \cdot n \lg{n} &\leq \lg{n!}, &\forall n \geq 4 = n_0 \\
\end{align*}
故$\lg{(n!)}=\Theta(n \lg{n})$。\par
下证$n!=\omega (2^n)$且$n!=o(n^n)$:
\begin{proof}
\begin{align*}
    & \lim_{n \to \infty}{\frac{n!}{n^n}} \\
    =& \lim_{n \to \infty}{\prod_{i=1}^n{\frac{i}{n}}} \\
    =& 0 \\
    \\
    & \lim_{n \to \infty}{\frac{n!}{2^n}} \\
    =& \lim_{n \to \infty}{\prod_{i=1}^n{\frac{i}{2}}} \\
    =& \infty \\
\end{align*}
\end{proof}

\section{Exercise 4.3-2 (With the substitution method) }
证明:$T(n)=T(\lceil n/2 \rceil)+1$的解为$O(\lg{n})$。
\\

\paragraph{解}
\begin{proof}
要证对某个常数$c > 0$,$T(n) \leq c \cdot \lg{n}$成立。假设此上界对所有正数$m < n$都成立,特别是对于$m = \lceil n/2 \rceil$,有$T(\lceil n/2 \rceil) \leq c \cdot \lg{(\lceil n/2 \rceil)}$,将其代入递归式,得到
\begin{align*}
    T(n) \leq & c \cdot \lg{\left(\left\lceil \frac{n}{2} \right\rceil \right)} + 1 \\
    < & c \cdot \lg{\left( \frac{n}{2} + 1 \right)} + 1 \\
    =& c \lg{( n + 2 )} - c \lg{2} + 1 \\
    =& c \lg{( n + 2 )} - c + 1 \\
    \leq & c \lg{n}
\end{align*}
其中,为使最后一步成立,应有
\begin{align*}
    && c \lg{n} &\geq c \lg{(n + 2)} - c + 1 & \\
    \Rightarrow && -1 &\geq c \left(\lg{(n + 2)} - \lg{n} - 1 \right) & \\
    \Rightarrow && c &\geq \frac{1}{1 - \lg{\dfrac{n + 2}{n} }}, & n > 2 \\
    \Rightarrow && c &\geq \frac{1}{1 - \lg{\dfrac{5}{3}}} = \frac{1}{\lg{\dfrac{6}{5}}}, &  n > 2 \\
\end{align*}
取$c = \displaystyle \frac{1}{\lg{\dfrac{6}{5}}}$,假设$T(1) = 1$,由递归式可得
\begin{align*}
    T(2) =& T(\lceil 2/2 \rceil) + 1 \\
    =& T(1) + 1 \\
    =& 2 \\
    <& \frac{1}{\lg{\dfrac{6}{5}}} \\
    =& c \lg{2} \\
\end{align*}
因此$c$的取值对于$n \geq 2$都是合适的。
\end{proof}

\section{Exercise 4.4-8}
对递归式$T(n)=T(n-a)+T(a)+cn$。利用递归树给出一个渐进紧确解,其中$a \geq 1$和$c>0$是常数。
\\

\paragraph{解}
递归树如图\ref{fig:1}所示,
\begin{figure}
    \centering
    \begin{tikzpicture}[level/.style={sibling distance=60mm/#1}, scale = 0.8, transform shape]
    \node (a) {$cn$}
        child {node  (b1) {$c(n-a)$}
            child {node (c1) {$c(n-2a)$}
                child {node (d1) {$\vdots$}
                    child {node  (e1) {$ca$}}
                    child {node  (e2) {$ca$}}
                }
                child {node (d2) {$ca$}}
            }
            child {node (c2) {$ca$}}
        }
        child {node  (b2) {$ca$}
            child [grow=right] {node {$=$} edge from parent[draw=none]
                child [grow=right] {node (z1) {$cn$} edge from parent[draw=none]
                    child [grow=up] {node (z0) {$cn$} edge from parent[draw=none]}
                    child [grow=down] {node (z2) {$c(n-a)$} edge from parent[draw=none]
                        child [grow=down] {node (z3) {$\vdots$} edge from parent[draw=none]
                            child [grow=down] {node (z4) {$c \cdot 2a$} edge from parent[draw=none]
                                child [grow=down] {node (z5) {$3c(n-a)$} edge from parent[draw=none]}
                            }
                        }
                    }
                }
            }
        };
    \path (b1) -- (b2) node [midway] {$+$};
    \path (c1) -- (c2) node [midway] {$+$};
    \path (e1) -- (e2) node [midway] {$+$};
    \path (a) -- (z0) node [midway] {$=$};
    \path (c2) -- (z2) node [midway] {$=$};
    \path (d2) -- (z3) node [midway] {$\cdots$};
    \path (e2) -- (z4) node [midway] {$=$}
        child [grow=down] {node (y) {$cn+\displaystyle \sum_{i=0}^{n/a - 2}{c(n-i \cdot a)}$} edge from parent[draw=none]};
    \path (y) -- (z5) node [midway] {$=$};
    \end{tikzpicture}
    \caption{\label{fig:1}为递归式$T(n)=T(n-a)+T(a)+cn$构造递归树}
\end{figure}
则整棵树的总代价为$T(n) = 3c(n-a) = \Theta(n)$。
\\

\section{Exercise 4.5-1}
对下列递归式,使用主方法求出渐进紧确界。
\subparagraph{b}
$T(n)=2T(n/4)+\sqrt{n}$
\subparagraph{d}
$T(n)=2T(n/4)+n^2$
\\

\paragraph{解}
\subparagraph{b}
对于这个递归式,我们有$a=2$,$b=4$,$f(n)=\sqrt{n}$,因此$n^{\log_b{a}}=n^{\log_4{2}}=n^{1/2}$。由于$f(n)=\sqrt{n}=\Theta\left(n^{1/2}\right)$,因此可以应用主定理的情况2,从而得到解$T(n)=\Theta\left(n^{1/2} \lg{n}\right)$。
\subparagraph{d}
对于这个递归式,我们有$a=2$,$b=4$,$f(n)=n^2$,因此$n^{\log_b{a}}=n^{\log_4{2}}=n^{1/2}$。由于$f(n)=n^2=\Omega\left(n^{\log_4{2}+\varepsilon}\right)$,其中$\varepsilon = 3/2$,因此如果可以证明正则条件成立,即可应用主定理的情况3。当$n$足够大时,对于$c=1/8$
\begin{equation*}
    a f\left(\frac{n}{b}\right) = 2\left(\frac{n}{4}\right)^2 \leq \frac{1}{8}n^2 = c f(n)
\end{equation*}
因此,由情况3,递归式的解为$T(n) = \Theta\left(n^2\right)$。
\\

\FloatBarrier
\section{Exercise 4.5-4}
主方法能应用于递归式$T(n)=4T(n/2)+n^2\lg{n}$吗?请说明为什么可以或者为什么不可以。给出这个递归式的一个渐进上界。
\\

\paragraph{解}
不可以应用主定理。\par
对于这个递归式,有$a=4$,$b=2$,$f(n)=n^2\lg{n}$,因此$n^{\log_b{a}} = n^{\log_2{4}} = n^2$。对任意常数$\varepsilon > 0$
\begin{equation*}
    \frac{f(n)}{n^{\log_b{a}}} = \frac{n^2\lg{n}}{n^2} = \lg{n} = o(n^{\varepsilon})
\end{equation*}
因此主定理的三种情况都不成立。\par
下面用构造递归树(如图\ref{fig:2}所示)并用代入法证明的方法给出这个递归式的渐进上界。
\begin{figure}
    \centering
    \begin{tikzpicture}[
        level 1/.style = {sibling distance = 100mm, level distance = 120mm},
        level 2/.style = {sibling distance = 25mm, level distance = 80mm},
        level 3/.style = {level distance = 80mm},
        scale=0.2,
        transform shape]
    \node [scale=5] (a) {$n^2\lg{n}$}
        child {node [scale=3] (b1) {$\left(\dfrac{n}{2}\right)^2 \lg{\dfrac{n}{2}}$}
            child {node (c1) {$\left(\dfrac{n}{4}\right)^2 \lg{\dfrac{n}{4}}$}
                child [grow = down] {node [scale=3] {$\vdots$} edge from parent[draw=none]}
                child [grow = left] {node [scale=5] (depth) {$\log_2{n} + 1$} edge from parent[draw = none]
                    child [grow = up] {node (arrow1) {} edge from parent[->]}
                    child [grow = down] {node (arrow1) {}edge from parent[->]}
                }
            }
            child {node (c2) {$\left(\dfrac{n}{4}\right)^2 \lg{\dfrac{n}{4}}$}
                child {node [scale=3] {$\vdots$} edge from parent[draw=none]}
            }
            child {node (c3) {$\left(\dfrac{n}{4}\right)^2 \lg{\dfrac{n}{4}}$}
                child {node [scale=3] {$\vdots$} edge from parent[draw=none]}
            }
            child {node (c4) {$\left(\dfrac{n}{4}\right)^2 \lg{\dfrac{n}{4}}$}
                child {node [scale=3] {$\vdots$} edge from parent[draw=none]}
            }
        }
        child {node [scale=3] (b2) {$\left(\dfrac{n}{2}\right)^2 \lg{\dfrac{n}{2}}$}
            child {node (c5) {$\left(\dfrac{n}{4}\right)^2 \lg{\dfrac{n}{4}}$}
                child {node [scale=3] {$\vdots$} edge from parent[draw=none]}
            }
            child {node (c6) {$\left(\dfrac{n}{4}\right)^2 \lg{\dfrac{n}{4}}$}
                child {node [scale=3]{$\vdots$} edge from parent[draw=none]}
            }
            child {node (c7) {$\left(\dfrac{n}{4}\right)^2 \lg{\dfrac{n}{4}}$}
                child {node [scale=3] {$\vdots$} edge from parent[draw=none]}
            }
            child {node (c8) {$\left(\dfrac{n}{4}\right)^2 \lg{\dfrac{n}{4}}$}
                child {node [scale=3] {$\vdots$} edge from parent[draw=none]}
            }
        }
        child {node [scale=3] (b3) {$\left(\dfrac{n}{2}\right)^2 \lg{\dfrac{n}{2}}$}
            child {node (c9) {$\left(\dfrac{n}{4}\right)^2 \lg{\dfrac{n}{4}}$}
                child {node [scale=3] {$\vdots$} edge from parent[draw=none]}
            }
            child {node (c10) {$\left(\dfrac{n}{4}\right)^2 \lg{\dfrac{n}{4}}$}
                child {node [scale=3]{$\vdots$} edge from parent[draw=none]}
            }
            child {node (c11) {$\left(\dfrac{n}{4}\right)^2 \lg{\dfrac{n}{4}}$}
                child {node [scale=3] {$\vdots$} edge from parent[draw=none]}
            }
            child {node (c12) {$\left(\dfrac{n}{4}\right)^2 \lg{\dfrac{n}{4}}$}
                child {node [scale=3] {$\vdots$} edge from parent[draw=none]}
            }
        }
        child {node [scale=3] (b4) {$\left(\dfrac{n}{2}\right)^2 \lg{\dfrac{n}{2}}$}
            child {node (c13) {$\left(\dfrac{n}{4}\right)^2 \lg{\dfrac{n}{4}}$}
                child {node [scale=3] {$\vdots$} edge from parent[draw=none]}
            }
            child {node (c14) {$\left(\dfrac{n}{4}\right)^2 \lg{\dfrac{n}{4}}$}
                child {node [scale=3] {$\vdots$} edge from parent[draw=none]}
            }
            child {node (c15) {$\left(\dfrac{n}{4}\right)^2 \lg{\dfrac{n}{4}}$}
                child {node [scale=3] {$\vdots$} edge from parent[draw=none]}
            }
            child {node (c16) {$\left(\dfrac{n}{4}\right)^2 \lg{\dfrac{n}{4}}$}
                child [grow=down]{node [scale=4] {$\vdots$} edge from parent[draw=none]}
                child [grow=right] {node [scale=4] (z2) {$n^2(\lg{n} - 2)$} edge from parent[draw=none]
                        [level distance=80mm]
                        child [grow=down] {node [scale=4] (z3) {$\vdots$} edge from parent[draw=none]
                            child [grow=down] {node [scale=4] (z4) {$\Theta \left(n^{\log_2{4}}\right)$} edge from parent[draw=none]
                                child [grow=down] {node [scale=5] (z5) {$O \left(n^2 \lg^2{n}\right)$} edge from parent[draw=none]}
                                child [grow=left] {
                                    [level distance=18mm]
                                    node (y1) {$T(1)$} edge from parent[draw=none]
                                        child [grow=left] {node (y2) {$T(1)$} edge from parent[draw=none]
                                            child [grow=left] {node (y3) {$T(1)$} edge from parent[draw=none]
                                                child [grow=left] {node (y4) {$T(1)$} edge from parent[draw=none]
                                                    child [grow=left] {node (y5) {$T(1)$} edge from parent[draw=none]
                                                        child [grow=left] {node (y6) {$T(1)$} edge from parent[draw=none]
                                                            child [grow=left] {node (y7) {$T(1)$} edge from parent[draw=none]
                                                                child [grow=left] {node (y8) {$T(1)$} edge from parent[draw=none]
                                                                    child [grow=left] {node (y9) {$T(1)$} edge from parent[draw=none]
                                                                        child [grow=left] {node (y10) {$T(1)$} edge from parent[draw=none]
                                                                            child [grow=left] {node (y11) {$\cdots$} edge from parent[draw=none]
                                                                                child [grow=left] {node (y12) {$\cdots$} edge from parent[draw=none]
                                                                                    child [grow=left] {node (y13) {$T(1)$} edge from parent[draw=none]
                                                                                        child [grow=left] {node (y14) {$T(1)$} edge from parent[draw=none]
                                                                                            child [grow=left] {node (y15) {$T(1)$} edge from parent[draw=none]
                                                                                                child [grow=left] {node (y16) {$T(1)$} edge from parent[draw=none]
                                                                                                    child [grow=left] {node (y17) {$T(1)$} edge from parent[draw=none]
                                                                                                        child [grow=left] {node (y18) {$T(1)$} edge from parent[draw=none]
                                                                                                            child [grow=left] {node (y19) {$T(1)$} edge from parent[draw=none]
                                                                                                                child [grow=left] {node (y20) {$T(1)$} edge from parent[draw=none]
                                                                                                                    child [grow=left] {node (y21) {$T(1)$} edge from parent[draw=none]
                                                                                                                        child [grow=left] {node (y22) {$T(1)$} edge from parent[draw=none]}
                                                                                            }
                                                                                                                }
                                                                                            }
                                                                                                        }
                                                                                            }
                                                                                                }
                                                                                            }
                                                                                        }
                                                                                    }
                                                                                }
                                                                            }
                                                                        }
                                                                    }
                                                                }
                                                            }
                                                        }
                                                    }
                                                }
                                            }
                                        }
                                    }
                            }
                        }
                        child [grow=up] {node [scale=4] (z1) {$n^2(\lg{n} - 1)$} edge from parent[draw=none]
                            [level distance=120mm]
                            child [grow=up] {node [scale=4] (z0) {$n^2\lg{n}$} edge from parent[draw=none]}
                        }
                }
            }
        };
    \end{tikzpicture}
    \caption{为递归式$T(n)=4T(n/2)+n^2\lg{n}$构造递归树}
    \label{fig:2}
\end{figure}
整棵递归树的代价为
\begin{align*}
    T(n) =& \sum_{i=0}^{\log_2{n}}{n^2(\lg{n}-i)} \\
    =& \sum_{i=0}^{\lg{n} - 1}{n^2\lg{n}} - n^2\sum_{i=1}^{\lg{n}}{i} \\
    =& n^2 \lg^2{n} - n^2\frac{\lg{n}(\lg{n}+1)}{2} \\
    =& \frac{1}{2} n^2 \lg^2{n} - \frac{1}{2} n^2 \lg{n} \\
    =& O(n^2\lg^2{n})
\end{align*}
下证递归式$T(n)=4T(n/2)+n^2\lg{n} = O(n^2\lg^2{n})$,即证恰当选择常数$c>0$,可有$T(n) \leq cn^2\lg^2{n}$。首先假定此上界对所有正数$m<n$都成立,特别是对于$m = n / 2$,有$T(n/2) \leq c (n/2)^2 \lg^2{(n/2)}$。将其代入递归式,得到
\begin{align*}
    T(n) \leq& 4c \left( \dfrac{n}{2} \right)^2 \lg^2{\dfrac{n}{2}} + n^2\lg{n}\\
    =& cn^2(\lg{n} - 1)^2 + n^2\lg{n}\\
    \leq& cn^2\lg^2{n}
\end{align*}
其中,为使最后一步成立,应有
\begin{align*}
    && cn^2(\lg{n} - 1)^2 + n^2\lg{n} &\leq cn^2\lg^2{n} & \\
    \Rightarrow && c(1 - 2\lg{n}) + n^2\lg{n} &\leq 0 & \\
    \Rightarrow && c &\geq \frac{\lg{n}}{2\lg{n} - 1}, & n \geq 2 \\
    \Rightarrow && c &\geq \frac{\lg{2}}{2\lg{2} - 1} = 1, & n \geq 2
\end{align*}
即$n \geq 2 $时,只要$c \geq 1$,前式的最后一步都会成立,猜测正确。
\FloatBarrier
\end{document}