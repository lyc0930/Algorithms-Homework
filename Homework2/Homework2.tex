\documentclass{article}
\usepackage[UTF8]{ctex}
\usepackage[T1]{fontenc}
\usepackage[utf8]{inputenc}
\usepackage{float}
\usepackage{placeins}
\usepackage{latexsym}
\usepackage{algorithm}
\usepackage{algorithmic}
\usepackage{amsmath}
\usepackage{amsthm}
\title{Homework 1}
\author{PB17000297 罗晏宸}
\date{September 13 2019}

\begin{document}

\maketitle

\section{Exercise 3.1-4}
$2^{n+1}=O(2^n)$成立吗?$2^{2n}=O(2^n)$成立吗?
\\

\paragraph{解} 
\\

\section{Exercise 3.2-3}
证明等式
\begin{equation}
    \lg{(n!)}=\Theta(n \lg{n}) \tag{3.19}
\end{equation}
并证明$n!=\omega (2^n)$且$n!=o(n^n)$。
\\

\paragraph{解}
\\

\section{Exercise 4.3-2 (Use substitution method) }
证明:$T(n)=T(\lceil n/2 \rceil)+1$的解为$O(\lg{n})$。
\\

\paragraph{解}
\\

\section{Exercise 4.4-8}
对递归式$T(n)=T(n-a)+T(a)+cn$。利用递归树给出一个渐进紧确解,其中$a \geq 1$和$c>0$是常数。
\\

\paragraph{解}
\\

\section{Exercise 4.5-1}
对下列递归式,使用主方法求出渐进紧确界。
\subparagraph{b}
$T(n)=2T(n/4)+\sqrt{n}$
\subparagraph{d}
$T(n)=2T(n/4)+n^2$
\\

\paragraph{解}
\\

\section{Exercise 4.5-4}
主方法能应用于递归式$T(n)=4T(n/2)+n^2\lg{n}$吗?请说明为什么可以或者为什么不可以。给出这个递归式的一个渐进上界。
\\

\paragraph{解}
\\
\end{document}