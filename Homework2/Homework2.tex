\documentclass{article}
\usepackage[UTF8]{ctex}
\usepackage[T1]{fontenc}
\usepackage[utf8]{inputenc}
\usepackage{float}
\usepackage{placeins}
\usepackage{latexsym}
\usepackage{algorithm}
\usepackage{algorithmic}
\usepackage{amsmath}
\usepackage{amsthm}
\title{Homework 1}
\author{PB17000297 罗晏宸}
\date{September 13 2019}

\begin{document}

\maketitle

\section{Exercise 3.1-4}
$2^{n+1}=O(2^n)$成立吗?$2^{2n}=O(2^n)$成立吗?
\\

\paragraph{解} 
前者成立,但后者不成立。\par
取$c=2$与$n_0=1$,有
\begin{equation*}
    0 \leq 2^{n+1} = 2 \cdot 2^n \leq c \cdot 2^n, \qquad \forall n \geq n_0 = 1
\end{equation*}
因此$2^{n+1}=O(2^n)$。\par
假设存在常数$c$与$n_0$使得
\begin{equation*}
    0 \leq 2^{2n} \leq c \cdot 2^n, \qquad \forall n \geq n_0
\end{equation*}
成立,则有
\begin{align*}
    && 0 \leq 2^{2n} &\leq c \cdot 2^n, & \forall n \geq n_0 \\
    \Rightarrow && 2^n \cdot 2^n &\leq c \cdot 2^n, &  \forall n \geq n_0 \\
    \Rightarrow && 2^n &\leq c, & \forall n \geq n_0
\end{align*}
这与$c$是常数矛盾!因此假设不成立,$2^{2n} \neq O(2^n)$。
\\

\section{Exercise 3.2-3}
证明等式
\begin{equation}
    \lg{(n!)}=\Theta(n \lg{n}) \tag{3.19}
\end{equation}
并证明$n!=\omega (2^n)$且$n!=o(n^n)$。
\\

\paragraph{解}
取常数$c_1 = 1 - \dfrac{\lg{e}}{2}$、$c_2 = 1$和$n_0 = 4$,有
\begin{align*}
    && n! &\leq n^n, & \forall n \geq 1 \\
    \Rightarrow && \lg{n!} &\leq \lg{n^n}, & \forall n \geq 1 \\
    \Rightarrow && \lg{n!} &\leq n \lg{n} = 1 \cdot n \lg{n}, & \forall n \geq 1 \\
    \Rightarrow && \lg{n!} &\leq c_2 \cdot n \lg{n}, & \forall n \geq n_0 > 1 \\
    \\
    && c_1 = 1 - \frac{\lg{e}}{2} &\leq 1 - \dfrac{\lg{e}}{n}, & \forall n \geq 4 \\
    \Rightarrow && \frac{\lg{e}}{\lg{n}} &\leq 1 - c_1, & \forall n \geq 4 \\
    \Rightarrow && \lg{e} &\leq (1-c_1)\lg{n}, & \forall n \geq 4 \\
    \Rightarrow && n\lg{e} &\leq n \lg{n} - c_1 \cdot n\lg{n}, & \forall n \geq 4 \\
    \Rightarrow && c_1 \cdot n \lg{n} &\leq n\lg{n}-n\lg{e}, & \forall n \geq 4 \\
    \Rightarrow && c_1 \cdot n \lg{n} &\leq \lg{\left(\frac{n}{e}\right)^n}, & \forall n \geq 4 \\
    \Rightarrow && c_1 \cdot n \lg{n} &\leq \lg{\left[\sqrt{2 \pi n}\left(\frac{n}{e}\right)^n\right]}, & \forall n \geq 4 \\
    \Rightarrow && c_1 \cdot n \lg{n} &\leq \lg{\left[\sqrt{2 \pi n}\left(\frac{n}{e}\right)^n \left(1+\Theta \left(\frac{1}{n}\right) \right)\right]}, & \forall n \geq 4 \\
    \Rightarrow && c_1 \cdot n \lg{n} &\leq \lg{n!}, & \forall n \geq 4 = n_0 \\
    \Right
\end{align*}
故$\lg{(n!)}=\Theta(n \lg{n})$。\par
下证$n!=\omega (2^n)$且$n!=o(n^n)$:
\begin{proof}
\begin{align*}
    & \lim_{n \to \infty}{\frac{n!}{n^n}} \\
    =& \lim_{n \to \infty}{\prod_{i=1}^n{\frac{i}{n}}} \\
    =& 0 \\
    \\
    & \lim_{n \to \infty}{\frac{n!}{2^n}} \\
    =& \lim_{n \to \infty}{\prod_{i=1}^n{\frac{i}{2}}} \\
    =& \infty \\
\end{align*}
\end{proof}
\\

\section{Exercise 4.3-2 (Use substitution method) }
证明:$T(n)=T(\lceil n/2 \rceil)+1$的解为$O(\lg{n})$。
\\

\paragraph{解}
\\

\section{Exercise 4.4-8}
对递归式$T(n)=T(n-a)+T(a)+cn$。利用递归树给出一个渐进紧确解,其中$a \geq 1$和$c>0$是常数。
\\

\paragraph{解}
\\

\section{Exercise 4.5-1}
对下列递归式,使用主方法求出渐进紧确界。
\subparagraph{b}
$T(n)=2T(n/4)+\sqrt{n}$
\subparagraph{d}
$T(n)=2T(n/4)+n^2$
\\

\paragraph{解}
\\

\section{Exercise 4.5-4}
主方法能应用于递归式$T(n)=4T(n/2)+n^2\lg{n}$吗?请说明为什么可以或者为什么不可以。给出这个递归式的一个渐进上界。
\\

\paragraph{解}
\\
\end{document}