\documentclass{article}
\usepackage[UTF8]{ctex}
\usepackage[T1]{fontenc}
\usepackage[utf8]{inputenc}
\usepackage{latexsym}
\usepackage{amsmath}
\usepackage{amsthm}
\usepackage{amssymb}
\usepackage{hyperref}
\usepackage{clrscode3e}
\usepackage{xcolor}

\newcommand{\SetHrefColor}[1]{
	\hypersetup{urlcolor=#1}
}

\hypersetup{
	colorlinks=true,
	urlcolor=black,
}

\title{Homework 8}
\author{PB17000297 罗晏宸}
\date{November 24 2019}



\begin{document}

\maketitle

\section{Exercise 17.1-3}
假定我们对一个数据结构执行一个由 $n$ 个操作组成的操作序列,当 $i$ 严格为 $2$ 的幂时,第 $i$ 个操作的代价为 $i$,否则代价为 $1$。使用聚合分析确定每个操作的摊还代价。

\paragraph{解}
在执行的$n$个操作中,有至多$\lceil \lg{n} \rceil$个操作的次序是$2$的幂,这些操作的次序(即代价)如下
\begin{equation*}
	1, 2, 4, 8, \cdots , 2^{\lceil \lg{n} \rceil}
\end{equation*}
$n$个操作的总代价为
\begin{align*}
	T =& \sum_{k = 0}^{\lceil \lg{n} \rceil}{2^k} + (n - \lceil \lg{n} \rceil) \times 1 \\
	=& 2^{\lceil \lg{n} \rceil + 1} - 1 + (n - \lceil \lg{n} \rceil) \\
	\leq & 2^{\lg{n} + 2} + n - \lg{n} \\
	=& 3\lg{n} + n
\end{align*}
因此每个操作的摊还代价是$O\left(\dfrac{3\lg{n}+n}{n}\right) = O(1)$的。

\section{Exercise 17.2-2}
用核算法重做第一题。

\paragraph{解}
当操作次序是$2$的幂时,为其赋$4$的摊还代价,否则为其赋$2$的摊还代价,则每一个不为$2$的幂的操作均会提供$1$的信用以支付差额,对于一个$n$个操作组成的操作序列,有
\begin{align*}
	& 4 \times \lceil \lg{n} \rceil + 2 \times (n - \lceil \lg{n} \rceil) \\
	=& 2 \times \lceil \lg{n} \rceil + 2n \\
	\leq & 3 \lg{n} + n \\
	\leq & \sum_{k = 0}^{\lceil \lg{n} \rceil}{2^k} + (n - \lceil \lg{n} \rceil) \times 1
\end{align*}
每个操作的摊还代价都是常数,因此都是$O(1)$的。

\section{Exercise 17.3-2}
使用势能法重做第一题。

\paragraph{解}
定义势函数$\Phi(D_i)$
\begin{equation*}
	\Phi(D_i) = \left\{
	\begin{aligned}
		&0, &&i = 0 \\
		&1, &&i = 1 \\
		&\lg{i} + 1, &&i = 2^{\lfloor \lg{i} \rfloor} \\
		&\lfloor \lg{i} \rfloor + k, &&i = 2^{\lfloor \lg{i} \rfloor} + k \\
	\end{aligned}
	\right.
\end{equation*}
则有对任意的$i$,$\Phi(D_i) \geq 0$,且
\begin{equation*}
	\Phi(D_i) - \Phi(D_{i - 1}) = \left\{
	\begin{aligned}
		&1 - i, &&i = 2^{\lfloor \lg{i} \rfloor} \\
		&1, &&i = 2^{\lfloor \lg{i} \rfloor} + k \\
	\end{aligned}
	\right.
\end{equation*}
故有
\begin{align*}
	&\sum_{i = 1}^{n}{ \hat{c}_i } \\
	=&\sum_{i = 1}^{n}{1} \\
	=&n
\end{align*}
因此每个操作的摊还代价是$O(1)$。

\section{Problem 30-3 Multidimensional fast Fourier transform}
我们可以将一维离散傅里叶变换推广到 $d$ 维上。这时输入是一个 $d$ 维的数组 $A = (a_{j_1,j_2,\cdots,j_d})$,维数分别为 $n_1,n_2,\cdots,n_d$,其中 $n_1 n_2 \cdots n_d = n$。定义 $d$ 维离散傅里叶变换如下:
\begin{equation*}
	y_{k_1, k_2, \cdots, k_d} = \sum_{j_1 = 0}^{n_1 - 1}\sum_{j_2 = 0}^{n_2 - 1} \cdots \sum_{j_d = 0}^{n_d - 1}{a_{j_1,j_2,\cdots,j_d} \omega_{n_1}^{j_1 k_1}\omega_{n_2}^{j_2 k_2} \cdots \omega_{n_d}^{j_d k_d}}
\end{equation*}
其中 $0 \leq k_1 < n_1,0 \leq k_2 < n_2, \cdots ,0 \leq k_d < n_d$。
\subparagraph{a}证明:我们可以依次在每个维度上计算一维的 DFT 来计算一个 $d$ 维的 DFT。也就是说,首先沿着第 $1$ 维计算 $n / n_1$ 个独立的一维 DFT。然后,把沿着第 $1$ 维的 DFT 结果作为输入,我们计算沿着第 $2$ 维的 $n / n_2$ 个独立的一维 DFT。利用这个结果作为输入,我们计算沿着第三维的 $n / n_3$ 个独立的一维 DFT,如此下去,直到第 $d$ 维。
\subparagraph{b}证明:维度的次序并无影响,于是可以通过在 $d$ 个维度的任意顺序中计算一维 DFT 来计算一个 $d$ 维的 DFT。
\subparagraph{c}证明:如果采用计算快速傅里叶变换计算每个一维的 DFT,那么计算一个 $d$ 维的 DFT 的总时间是 $O(n\lg{n})$,与 $d$ 无关。

\paragraph{解}
\subparagraph{a}
可以交换求和号的顺序
\begin{align*}
	&y_{k_1, k_2, \cdots, k_d} \\
	=& \sum_{j_1 = 0}^{n_1 - 1}\sum_{j_2 = 0}^{n_2 - 1} \cdots \sum_{j_d = 0}^{n_d - 1}{a_{j_1,j_2,\cdots,j_d} \omega_{n_1}^{j_1 k_1}\omega_{n_2}^{j_2 k_2} \cdots \omega_{n_d}^{j_d k_d}} \\
	=& \sum_{j_d = 0}^{n_d - 1} \cdots \sum_{j_2 = 0}^{n_2 - 1} {\left( \sum_{j_1 = 0}^{n_1 - 1}{a_{j_1,j_2,\cdots,j_d} \omega_{n_1}^{j_1 k_1}} \right) \omega_{n_2}^{j_2 k_2} \cdots \omega_{n_d}^{j_d k_d}}
\end{align*}
故可以先计算$n/n_1$个独立的一维DFT:$\displaystyle \sum_{j_1 = 0}^{n_1 - 1}{a_{j_1,j_2,\cdots,j_d} \omega_{n_1}^{j_1 k_1}}$,再将其结果(记为$a_{j_1,j_2,\cdots,j_d}^{(1)}$)作为输入继续计算
\begin{align*}
	\sum_{j_d = 0}^{n_d - 1} \cdots \sum_{j_3 = 0}^{n_3 - 1} {\left( \sum_{j_2 = 0}^{n_2 - 1}{a_{j_1,j_2,\cdots,j_d}^{(1)} \omega_{n_2}^{j_2 k_2}} \right) \omega_{n_3}^{j_3 k_3} \cdots \omega_{n_d}^{j_d k_d}}
\end{align*}
直到第$d$维
\begin{align*}
	\sum_{j_d = 0}^{n_d - 1}{a_{j_1,j_2,\cdots,j_d}^{(d - 1)} \omega_{n_d}^{j_d k_d}}
\end{align*}

\subparagraph{b}
如a中所述的计算过程实际上由于求和号与乘号的交换性,可以任意调换顺序。

\subparagraph{c}
对于a中的计算过程,沿着第 $k$ 维的计算实际需要计算$\displaystyle n \Bigg/ \prod_{i \leq k}{n_k} $个独立的DFT,因此消耗 $O\left( \lg{n_k}  \cdot \displaystyle n \Bigg/ \prod_{i \leq k}{n_k} \right)$ 的时间,总的时间
\begin{align*}
	T =& \sum_{k = 1}^d{\frac{n}{\displaystyle\prod_{i \leq k}{n_k}} \cdot \lg{n_k}} \\
	\leq& \sum_{k = 1}^d{\frac{n}{\displaystyle\prod_{i \leq k}{n_k}} \cdot \lg{n}} \\
	\leq& \sum_{k = 1}^d{\frac{n}{\displaystyle\prod_{i \leq k}{2}} \cdot \lg{n}} \\
	\leq& n \lg{n} \cdot \sum_{k = 1}^d{\frac{1}{2}} \\
	= O(n \lg{n})
\end{align*}
即,与$d$无关
\end{document}