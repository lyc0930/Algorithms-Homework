\documentclass{article}
\usepackage[UTF8]{ctex}
\usepackage[T1]{fontenc}
\usepackage[utf8]{inputenc}
\usepackage{latexsym}
\usepackage{amsmath}
\usepackage{amsthm}
\usepackage{amssymb}
\usepackage{hyperref}
\usepackage{clrscode3e}
\usepackage{xcolor}

\newcommand{\SetHrefColor}[1]{
	\hypersetup{urlcolor=#1}
}

\hypersetup{
	colorlinks=true,
	urlcolor=black,
}

\title{Homework 8}
\author{PB17000297 罗晏宸}
\date{November 24 2019}



\begin{document}

\maketitle

\section{Exercise 17.1-3}
假定我们对一个数据结构执行一个由 $n$ 个操作组成的操作序列,当 $i$ 严格为 $2$ 的幂时,第 $i$ 个操作的代价为 $i$,否则代价为 $1$。使用聚合分析确定每个操作的摊还代价。

\paragraph{解}

\section{Exercise 17.2-2}
用核算法重做第一题。

\paragraph{解}

\section{Exercise 17.3-2}
使用势能法重做第一题。

\paragraph{解}


\section{Problem 30-3 Multidimensional fast Fourier transform}
我们可以将一维离散傅里叶变换推广到 $d$ 维上。这时输入是一个 $d$ 维的数组 $A = (a_{j_1,j_2,\cdots,j_d})$,维数分别为 $n_1,n_2,\cdots,n_d$,其中 $n_1 n_2 \cdots n_d = n$。定义 $d$ 维离散傅里叶变换如下:
\begin{equation*}
	y_{k_1, k_2, \cdots, k_d} = \sum_{j_1 = 0}^{n_1 - 1}\sum_{j_2 = 0}^{n_2 - 1} \cdots \sum_{j_d = 0}^{n_d - 1}{a_{j_1,j_2,\cdots,j_d} \omega_{n_1}^{j_1 k_1}\omega_{n_2}^{j_2 k_2} \cdots \omega_{n_d}^{j_d k_d}}
\end{equation*}
其中 $0 \leq k_1 < n_1,0 \leq k_2 < n_2, \cdots ,0 \leq k_d < n_d$。
\subparagraph{a}证明:我们可以依次在每个维度上计算一维的 DFT 来计算一个 $d$ 维的 DFT。也就是说,首先沿着第 $1$ 维计算 $n / n_1$ 个独立的一维 DFT。然后,把沿着第 $1$ 维的 DFT 结果作为输入,我们计算沿着第 $2$ 维的 $n / n_2$ 个独立的一维 DFT。利用这个结果作为输入,我们计算沿着第三维的 $n / n_3$ 个独立的一维 DFT,如此下去,直到第 $d$ 维。
\subparagraph{b}证明:维度的次序并无影响,于是可以通过在 $d$ 个维度的任意顺序中计算一维 DFT 来计算一个 $d$ 维的 DFT。
\subparagraph{c}证明:如果采用计算快速傅里叶变换计算每个一维的 DFT,那么计算一个 $d$ 维的 DFT 的总时间是 $O(n\lg{n})$,与 $d$ 无关。

\paragraph{解}

\end{document}